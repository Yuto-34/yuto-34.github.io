\documentclass[uplatex]{jsarticle}
\usepackage[svgnames,table,xcdraw,dvipsnames]{xcolor}
	\newcommand{\ricardo}[1]{\colorbox{ForestGreen}{\color{white}   \textsf{\textbf{Ricardo}}} \textcolor{ForestGreen}{#1}}
\usepackage[top=15truemm,bottom=20truemm,left=20truemm,right=20truemm]{geometry}
% \usepackage[preskip=1cm]{sectionbreak}
\usepackage[dvipdfmx]{graphicx}
\usepackage[version=3]{mhchem} %<- 化学反応式とか簡単に書けるやつ texdoc mhchem で説明
\usepackage{amsmath, amssymb, amsfonts}
\usepackage{ascmac, fancybox}
	% \fboxrule=2\fboxrule
	% \shadowsize=0.05\shadowsize
\usepackage[nobreak]{cite} %<- 参考文献
\usepackage[at]{easylist}
\usepackage{tcolorbox}
	\tcbuselibrary{raster,skins}
	\tcbuselibrary{breakable}
\usepackage{listings, jvlisting}
	\renewcommand{\lstlistingname}{リスト}
\usepackage{pgfplots}
\usepackage{multicol}
\usepackage{multirow}
\usepackage{wrapfig}
\usepackage{physics}
\usepackage{siunitx}
\usepackage{comment}
\usepackage{bchart}
\usepackage{cases}
\usepackage{tikz}
	\usetikzlibrary{intersections, calc, arrows, patterns, datavisualization}
% \usepackage[active,tightpage]{preview}
	% \PreviewEnvironment{tikzpicture}
	% \RequirePackage{luatex85}
% \usepackage{luatexja}
\usepackage{here}
\usepackage{otf} %\ajRoman{1}でローマ数字が使えるようになる
\usepackage{url} %<- \url{https://google.co.jp} でurlを適切に挿入できる
\usepackage{bm}
\pgfplotsset{compat=1.17}

%useable lang
%ABAP (R/2 4.3, R/2 5.0, R/3 3.1, R/3 4.6C, R/3 6.10), 
% ACM, ACMscript, ACSL, Ada (2005, 83, 95), Algol (60, 68), 
% Ant, Assembler (Motorola68k, x86masm), Awk (gnu, POSIX), 
% bash, Basic (Visual), C (ANSI, Handel, Objective, Sharp), 
% C++ (ANSI, GNU, ISO, Visual) Caml (light, Objective), CIL, 
% Clean, Cobol (1974, 1985, ibm), Comal 80, 
% command.com (WinXP), Comsol, csh, Delphi, Eiffel, Elan, 
% erlang, Euphoria, Fortran (03, 08, 77, 90, 95), 
% GAP, GCL, Gnuplot, hansl, Haskell, HTML, IDL (empty, CORBA), 
% inform, Java (empty, AspectJ), JVMIS, ksh, 
% Lingo, Lisp (empty, Auto), LLVM, Logo, Lua (5.0, 5.1, 5.2), 
% make (empty, gnu), Mathematica (1.0, 3.0, 5.2), Matlab, 
% Mercury, MetaPost, Miranda, Mizar, ML, Modula-2, MuPAD, 
% NASTRAN, Oberon-2, OCL (decorative, OMG), Octave, Oz, 
% Pascal (Borland6, Standard, XSC), Perl, PHP, PL/I, Plasm, 
% PostScript, POV, Prolog, Promela, PSTricks, Python, R, Reduce, 
% Rexx, RSL, Ruby, S (empty, PLUS), SAS, Scilab, sh, SHELXL, 
% Simula (67, CII, DEC, IBM), SPARQL, SQL, tcl (empty, tk), 
% TeX (AlLaTeX, common, LaTeX, plain, primitive), VBScript, 
% Verilog, VHDL (empty, AMS), VRML (97), XML, XSLT

\lstset{
	language=C,
	basicstyle={\ttfamily\small}, % 書体の指定
	frame=tRBl, % フレームの指定
	framesep=20pt, % フレームと中身(コード)の間隔
	breaklines=true, % 行が長くなった場合の改行
	linewidth=12cm, % フレームの横幅
	lineskip=-0.5ex, % 行間の調整
	tabsize=4, % Tabを何文字幅にするかの指定
	numberstyle=\scriptsize, % 行番号の書体
	stepnumber=1, % 行番号をいくつとばしで表示するか.5行ごとに行番号を表示したいときなど
	numbers=left, % 行番号表示位置.デフォルト:none(行番号なし),他オプション:left,","right
	numbersep=10pt, % 行番号と本文の間隔.デフォルト:10pt
	backgroundcolor=\color[gray]{.97}, % 背景色
	identifierstyle={\small}, % キーワードでない文字の書体
	commentstyle={\small\ttfamily \color[rgb]{0,0.5,0}}, % コメントの書体
	keywordstyle={\small\bfseries \color[rgb]{1,0,0}}, % キーワード(int, ifなど)の書体
	ndkeywordstyle={\small \color[rgb]{1,1,0}}, % キーワードその2の書体
	stringstyle={\small\ttfamily \color[rgb]{0,0,1}}, % ""で囲まれたなどの"文字"の書体
	columns=[l]{fullflexible}, % 書体による列幅の違いを調整するか. オプション:",fixed,"flexible,spaceflexible,fullflexible
	% xrightmargin=0zw, % 
	% xleftmargin=3zw, % 
	morecomment=[l]{//}, % 行コメント
	morecomment = [s]{/*}{*/}, % 複数行コメント
	classoffset=0, % 関数名等の色の設定
	showstringspaces=false, % 半角スペースを表示するか(style に依存
}

%行数,列数指定用
\makeatletter
\def\mojiparline#1{
	\newcounter{mpl}
	\setcounter{mpl}{#1}
	\@tempdima=\linewidth
	\advance\@tempdima by-\value{mpl}zw
	\addtocounter{mpl}{-1}
	\divide\@tempdima by \value{mpl}
	\advance\kanjiskip by\@tempdima
	\advance\parindent by\@tempdima
}
\makeatother
\def\linesparpage#1{
	\baselineskip=\textheight
	\divide\baselineskip by #1
}

% \renewcommand{\thepart}{\arabic{part}}
% \renewcommand{\prepartname}{校正}
% \renewcommand{\postpartname}{稿}
%\renewcommand{\thesection}{\arabic{section}.}
%\renewcommand{\thesubsection}{(\arabic{subsection})}
%\renewcommand{\thesubsubsection}{{ $ >$}}
%\renewcommand{\labelenumi}{(\arabic{enumi})}
%\setcounter{subsection}{15}

%\fallingdotseq ニアリーイコール
%\qty(){}[] 括弧

\title{ よくでる微分方程式 }
\date{\today}
\author{ Yuto }

\begin{document}
% 一行あたり文字数の指定
% \mojiparline{40}
% 1ページあたり行数の指定
% \linesparpage{30}
\maketitle
%\tableofcontents
%\clearpage

\renewcommand{\thesection}{その\arabic{section}.}
\section{$\displaystyle m\dv{x}{t} = \lambda x(t)$}

\begin{align*}
	m\dv{x}{t} &= \lambda x \\
	\int\dfrac{\mathrm{d}x}{x} &= \int \dfrac{\lambda }{m} \mathrm{d}t \\
	\log x &= \dfrac{\lambda }{m}t + c \\
	x &= \exp\qty[\dfrac{\lambda }{m}t + c] \\
	&= e^c\exp\qty[\dfrac{\lambda }{m}t] \\
	&= C\exp\qty[\dfrac{\lambda }{m}t] \qq{($\because e^c = \mathrm{const}. = C$)}
\end{align*}

\section{$\displaystyle m\dv[2]{x}{t} = -kx(t)$}
\subsection{解}

$x(t) = f(t)$としたとき,$f''(t) = -\dfrac{k}{m}f(t)$になる$t$を考える.

ここで,
$$f(t) = \sin t\mbox{のとき,}f''(t) = -\sin t$$
$$f(t) = \cos t\mbox{のとき,}f''(t) = -\cos t$$
が成り立つことは有名である.

すなわち,これに任意係数が付いた
$$f(t) = D_1\sin t\mbox{のとき,}f''(t) = -\sin t$$
$$f(t) = D_2\cos t\mbox{のとき,}f''(t) = -\cos t$$
も当然だが成り立つ.

線形性より,
\[
	f(t) = D_1\sin t + D_2\cos t
	\Rightarrow
	f''(t) = -(D_1\sin t + D_2\cos t)
\]

したがって,
\[
	f(t) = D_1\sin \sqrt{\dfrac{k}{m}}t + D_2\sqrt{\dfrac{k}{m}}\cos t
	\Rightarrow
	f''(t) = -\dfrac{k}{m}(D_1\sin t + D_2\cos t)
\]

すなわち,$f''(t) = -\dfrac{k}{m}f(t)$の解は,$$f(t) = D_1\sin \sqrt{\dfrac{k}{m}}t + D_2\cos\sqrt{\dfrac{k}{m}} t$$

\subsection{別解(厳密な求め方)}
その1と同じように$x(t)$の解を$e^{\lambda t}$とした上で微分方程式を解く.
\begin{align*}
	\dv[2]{x}{t} &= -\dfrac{k}{m}x \\
	\dv[2]{t}(e^{\lambda t}) &= -\dfrac{k}{m}e^{\lambda t} \\
	\lambda^2e^{\lambda t} &= -\dfrac{k}{m}e^{\lambda t}  \\
	\lambda^2 &= -\dfrac{k}{m} \\
	\lambda &= \pm i \sqrt{\dfrac{k}{m}} \\
	\therefore x(t) &= \exp\qty[{i \sqrt{\frac{k}{m}}}], \exp[{-i \sqrt{\frac{k}{m}}}]
\end{align*}
このとき係数がついていても微分には影響しないので,$c_1,c_2 \in \mathbb{R}$とすると,
\begin{align*}
	x_1(t) &= c_1\exp[{i \sqrt{\frac{k}{m}}}] \\
	x_2(t) &= c_2\exp[{-i \sqrt{\frac{k}{m}}}]
\end{align*}
が成り立つ.

線形性より,
\begin{align*}
	x(t) &= x_1(t) + x_2(t) \\
	&= c_1\exp[{i \sqrt{\frac{k}{m}}}] + c_2\exp[{-i \sqrt{\frac{k}{m}}}]
\end{align*}
が言える.

ここで,オイラーの等式$e^{i\theta} = \cos \theta + i\sin\theta$を用いると,
\begin{align*}
	x(t) &= c_1\qty(\cos\sqrt{\dfrac{k}{m}} + i\sin\sqrt{\dfrac{k}{m}}) + c_2\qty(\cos\sqrt{\dfrac{k}{m}} + i\sin\sqrt{\dfrac{k}{m}}) \\
	&= (c_1 + c_2)\qty(\cos\sqrt{\dfrac{k}{m}}) + i(c_2 - c_1)\qty(\sin\sqrt{\dfrac{k}{m}})
\end{align*}
となる.
ここで$c_1 + c_2 = C_1, i(c_2 - c_1) = C_2$と再定義して,
\begin{align*}
	x(t) = C_1\qty(\cos\sqrt{\dfrac{k}{m}}) + C_2\qty(\sin\sqrt{\dfrac{k}{m}})
\end{align*}




\end{document}
% Ctrl + Alt + C